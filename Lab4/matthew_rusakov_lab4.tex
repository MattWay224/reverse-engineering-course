\usepackage[russian]{babel}
\usepackage{geometry}
\usepackage{graphicx}
\usepackage{hyperref}
\usepackage{listings}
\usepackage{color}
\usepackage[T1]{fontenc}
\usepackage[utf8]{inputenc}

\title{Анализ бинарного файла, скомпилированного afl компилятором - Lab4}
\author{Matthew Rusakov m.rusakov@innopolis.university SD-03}
\date{May 2025}

\begin{document}

\maketitle

\section{Предисловие}
Анализ проводился с использованием AFL++, поскольку он обнаружил больше уникальных сбоев за намного меньшее количество времени.

Также я добавил сантитайзеры ASan и UBSan для обнаружения большей части уязвимостей

\section{Найденные проблемы с AddressSanitizer}

\subsection{Heap-buffer-overflow}

\paragraph{Найдена проблема:}
Обнаружено чтение за пределами выделенной области памяти в процессе освобождения JSON-структур.

\paragraph{Местоположение:} Функция \texttt{json\_value\_free\_ex}, строка 200 исходного кода.

\paragraph{Анализ и причина:} Ошибка возникает при итеративном освобождении элементов объекта JSON. Проверка границ массива значений (\texttt{values}) выполняется некорректно из-за возможного несоответствия между декларируемым количеством элементов (\texttt{length}) и фактическим размером массива. Дефект проявляется при обработке специально сформированных входных данных, вызывающих повреждение метаданных объекта.

\subsection{Corrupted pointer}
\paragraph{Найдена проблема:} Выявлены случаи обращения по невалидным указателям во время освобождения памяти.

\paragraph{Местоположение:} Функция \texttt{json\_value\_free\_ex}, строка 179.

\paragraph{Анализ и причина:} Санитайзер ASan зафиксировал попытки обращения к памяти по адресам, не принадлежащим процессу. Анализ стека вызовов указывает на возможную порчу указателей в структуре JSON-объекта до момента его освобождения. Наиболее вероятная причина - отсутствие проверки целостности указателей перед операцией освобождения.

\subsection{Heap-buffer-overflow при парсинге}
\paragraph{Найдена проблема:} Обнаружен off-by-one error при обработке Unicode-последовательностей.

\paragraph{Местоположение:} Функция \texttt{json\_parse\_ex}, строка 338.

\paragraph{Анализ и причина:} Ошибка возникает при парсинге escape-последовательностей формата \textbackslash uXXXX вблизи конца входного буфера. Логика обработки предполагает наличие 4 байт после управляющего символа, но не выполняет корректную проверку доступности этих байт в буфере. Преинкремент указателя (\texttt{*++state.ptr}) приводит к чтению за границами выделенной памяти.

\section{Undefined Behavior Sanitize}
\subsection{Неопределённое поведение}
\paragraph{Найдена проблема:} Необрабатываемое исключение SIGILL на валидных входных данных.

\paragraph{Местоположение:} Инициализация парсера перед циклом в \texttt{json\_parse\_ex}.

\paragraph{Анализ и причина:} Санитайзер UBSan детектировал неопределённое поведение на этапе инициализации парсера. SIGILL (Illegal Instruction) свидетельствует о серьёзном повреждении состояния программы, вероятно вызванном: (1) некорректными операциями с указателями, (2) нарушением выравнивания данных, или (3) переполнением целочисленных типов. Точная причина требует дополнительного исследования с отладочной сборкой.

\section{Control Flow Integrity (CFI)}
\begin{itemize}
    \item Обнаруженные сбои проявлялись как \texttt{free(): invalid pointer}
    \item Анализ подтвердил ошибки управления памятью, а не нарушения потока управления
    \item Подчеркнул необходимость использования ASan для диагностики проблем памяти
\end{itemize}

\section{Выводы}
Фаззинг-тестирование выявило несколько уязвимостей безопасности памяти:

\begin{itemize}
    \item Критические проблемы управления кучей при очистке JSON-структур
    \item Ошибки проверки границ при обработке Unicode
    \item Уязвимости целостности указателей при операциях с памятью
\end{itemize}

    \begin{thebibliography}{1}
        \bibitem{githublink}
        GitHub Link: https://github.com/MattWay224/reverse-engineering-course
        В этом репозитории можно найти все лабы и информацию про каждое задание в каждой лабе
    \end{thebibliography}
\end{document}
