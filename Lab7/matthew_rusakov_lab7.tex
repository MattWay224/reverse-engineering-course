\usepackage[utf8]{inputenc}
\usepackage{geometry}
\usepackage[T1]{fontenc}
\usepackage[russian]{babel}

\title{Отчёт по результатам фаззинга с помощью libfuzzer - Lab7}
\author{Matthew Rusakov m.rusakov@innopolis.university SD-03}
\date{May 2025}

\begin{document}

    \maketitle


    \section{Инструмент}
    Для тестирования использовался \textbf{libFuzzer} с подключенными санитайзерами:
    \begin{itemize}
        \item AddressSanitizer (ASan) - для обнаружения ошибок работы с памятью
        \item UndefinedBehaviorSanitizer (UBSan) - для выявления неопределённого поведения
    \end{itemize}


    \section{Найденные проблемы}
    В ходе тестирования обнаружены следующие критические уязвимости:

    \begin{enumerate}
        \item \textbf{Heap-buffer-overflow} при обработке неполного токена "nul"
        \item \textbf{Heap-use-after-free} при обработке массива с пробелом
        \item \textbf{Bad-free} при обработке пустой строки
        \item \textbf{Undefined behavior} (применение ненулевого offset к null pointer)
        \item \textbf{Повторный heap-use-after-free} при обработке пустого массива "[]"
    \end{enumerate}


    \section{Анализ}

    \subsection{Heap-buffer-overflow}
    Проблема возникает при проверке токена "null".
    Парсер не проверяет границы входного буфера при последовательном чтении символов:
    \begin{verbatim}
        if ((end - state.ptr) < 3 || *(++ state.ptr) != 'u' ||
            *(++ state.ptr) != 'l' || *(++ state.ptr) != 'l')
    \end{verbatim}

    \subsection{Heap-use-after-free}
    Проблема возникает из-за неправильного управления памятью при обработке массивов:
    \begin{itemize}
        \item Память освобождается слишком рано (в функции new\_value)
        \item Последующий код продолжает использовать указатель на освобождённую память
        \item Особенно проявляется при обработке пробельных символов после закрывающей скобки
    \end{itemize}

    \subsection{Bad-free}
    Новая обнаруженная проблема связана с некорректным освобождением памяти при обработке пустых строк:
    \begin{itemize}
        \item Парсер пытается освободить память, которая не была выделена через malloc/calloc
        \item Адрес 0x502000004a2f находится прямо перед выделенной областью (1 байт до 1-байтной области)
        \item Проблема проявляется на входе "\"\"" (двойные кавычки без содержимого)
        \item Ошибка возникает в функции json\_value\_free\_ex при освобождении строковых значений
    \end{itemize}

    \subsection{Undefined behavior}
    \begin{itemize}
        \item Тип: Неопределённое поведение (UB)
        \item Местоположение: json\_fuzz.c:442:65
        \item Описание: Применение ненулевого смещения к нулевому указателю
        \item Условия возникновения:
        \begin{itemize}
            \item Проявляется при обработке специально сформированных JSON-структур
            \item Связано с неправильной проверкой указателей при работе с памятью
        \end{itemize}
        \item Риски:
        \begin{itemize}
            \item Непредсказуемое поведение программы
            \item Потенциальная уязвимость для атак типа "undefined behavior exploitation"
        \end{itemize}
    \end{itemize}

    \section{Выводы}
    Как и в предыдущих анализах получилось опознать проблемы с парсером: heap-buffer-overflow, heap-use-after-free, undefined behavior, однако в прошлых репортах не получалось обраружить bad-free


    \begin{thebibliography}{1}
        \bibitem{githublink}
        GitHub Link: https://github.com/MattWay224/reverse-engineering-course
        В этом репозитории можно найти все лабы и информацию про каждое задание в каждой лабе
    \end{thebibliography}
\end{document}