\usepackage[utf8]{inputenc}
\usepackage[T2]{fontenc}
\usepackage{graphicx}
\usepackage{hyperref}
\usepackage{listings}
\usepackage{xcolor}
\usepackage[russian]{babel}
\usepackage{geometry}

\title{Анализ инструментаризации бинарников - Lab5}
\author{Matthew Rusakov m.rusakov@innopolis.university SD-03}
\date{May 2025}

\begin{document}

    \maketitle

    \section*{Предисловие}

    Я собрал 4 бинарника:

    \begin{verbatim}
    clang main.c json_fuzz.c -o json_parser -lm
    afl-clang-fast main.c json_fuzz.c -o json_parser_afl -lm
    afl-clang-fast -fsanitize=address main.c json_fuzz.c -o json_parser_afl_asan -lm
    afl-clang-lto -fsanitize=memory main.c json_fuzz.c -o json_parser_msan -lm
    \end{verbatim}


    \section{Анализ бинарных файлов}

    \subsection{json\_parser (стандартная компиляция)}
    \textbf{\_}

    \textbf{Размер:} 25K

    \textbf{Инструментация:} Отсутствует

    \textbf{Системные вызовы:}
    \begin{itemize}
        \item Стандартная последовательность инициализации
        \item Минимальное использование \texttt{mmap/munmap}
        \item Нет доступа к \texttt{/proc/self/}
    \end{itemize}

    \textbf{Дизассемблирование:}
    \begin{itemize}
        \item Прямые вызовы стандартных библиотечных функций
        \item Отсутствие проверок памяти
        \item Чистый стековый фрейм без дополнительной instrumentation
    \end{itemize}

    \textbf{Вывод:} Базовый бинарник без средств диагностики

    \subsection{json\_parser\_afl (только AFL)}
    \textbf{\_}


    \textbf{Размер:} 137K (в 5.5x больше стандартного)

    \textbf{Инструментация:}
    \begin{itemize}
        \item Добавление coverage-инструментации AFL
        \item Вставка кода для отслеживания путей выполнения
    \end{itemize}

    \textbf{Системные вызовы:}
    \begin{itemize}
        \item Дополнительные \texttt{mmap} для shared memory
        \item Чтение/запись в \texttt{.cur\_input} для фаззинга
        \item Нет санитайзер-специфичных вызовов
    \end{itemize}

    \textbf{Дизассемблирование:}
    \begin{itemize}
        \item Вставки кода \texttt{\_\_afl\_} для сбора coverage
        \item Инструментация ветвлений (\texttt{\_\_afl\_prev\_loc})
        \item Сохранение контекста выполнения между итерациями
    \end{itemize}

    \textbf{Вывод:} Оптимизирован для сбора coverage-данных

    \subsection{json\_parser\_asan (AFL + ASan)}
    \textbf{\_}

    \textbf{Размер:} 1.6M (в 64x больше стандартного)

    \textbf{Инструментация:}
    \begin{itemize}
        \item Полная memory instrumentation
        \item Red zones вокруг всех объектов
        \item Shadow memory mapping
    \end{itemize}

    \textbf{Системные вызовы:}
    \begin{itemize}
        \item Частые \texttt{mmap/munmap} для shadow memory
        \item Постоянный мониторинг \texttt{/proc/self/maps}
        \item Пользовательские обработчики сигналов (SIGSEGV)
        \item \texttt{madvise} для оптимизации доступа к памяти
    \end{itemize}

    \textbf{Дизассемблирование:}
    \begin{itemize}
        \item Вызовы \texttt{\_\_asan\_*} перед каждой операцией с памятью
        \item Замена стандартных функций на \texttt{\_\_interceptor\_*} версии
        \item Вставки проверок стека (\texttt{\_\_asan\_stack\_malloc})
    \end{itemize}

    \textbf{Вывод:} Максимальная детекция ошибок памяти ценой производительности

    \subsection{json\_parser\_msan (AFL + MSan)}
    \textbf{\_}

    \textbf{Размер:} 1.3M (в 52x больше стандартного)

    \textbf{Инструментация:}
    \begin{itemize}
        \item Трассировка неинициализированной памяти
        \item Shadow memory для отслеживания битов инициализации
    \end{itemize}

    \textbf{Системные вызовы:}
    \begin{itemize}
        \item Схожи с ASan, но менее интенсивные
        \item Специфичные \texttt{msan\_} обработчики
        \item Меньше обращений к \texttt{/proc}
    \end{itemize}

    \textbf{Дизассемблирование:}
    \begin{itemize}
        \item Проверки \texttt{\_\_msan\_} при загрузке значений
        \item Инструментация перемещения памяти (\texttt{memcpy/memset})
        \item Отсутствие interceptors для файловых операций
    \end{itemize}

    \textbf{Вывод:} Специализирован на обнаружении use-of-uninitialized-memory


    \section{Выводы}
    \begin{itemize}
        \item \textbf{Рост размера:} ASan > MSan > AFL > Baseline
        \item \textbf{Накладные расходы:}
        \begin{itemize}
            \item ASan добавляет наибольшие overhead
            \item MSan требует меньше памяти чем ASan
            \item AFL минимально влияет на runtime-производительность
        \end{itemize}
        \item \textbf{При каких случаях что лучше использовать:}
        \begin{itemize}
            \item Для фаззинга: AFL + ASan (максимальный охват ошибок)
            \item Для production: Стандартная компиляция или AFL-only
            \item Для тестирования: MSan для сложных memory-багов
        \end{itemize}
    \end{itemize}
    \begin{thebibliography}{1}
        \bibitem{githublink}
        GitHub Link: https://github.com/MattWay224/reverse-engineering-course
        В этом репозитории можно найти все лабы и информацию про каждое задание в каждой лабе
    \end{thebibliography}
\end{document}