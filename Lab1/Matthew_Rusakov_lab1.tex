\documentclass{article}
\usepackage{graphicx} % Required for inserting images
\usepackage{tabularx}
\usepackage[T2A]{fontenc}
\usepackage[top=5cm,bottom=3cm,right=3cm,left=3cm]{geometry}
\usepackage[utf8]{inputenc}

\title{Анализ процессора i.MX RT1060}
\author{Матвей Русаков m.rusakov@innopolis.university SD-03}
\date{Апрель 2025}

\begin{document}

\maketitle



\section{Обзор}

i.MX RT1060 - это crossover микроконтроллер от NXP на базе одного ядра Arm Cortex-M7 с тактовой частотой до 600 МГц. Он интегрирует высокопроизводительные функции, такие как 1 МБ встроенной оперативной памяти, 32 КБ кэш-памяти инструкций, 32 КБ кэш-памяти данных и блок с плавающей точкой (VFPv5). Целевая область применения - промышленные HMI, управление двигателями, бытовая техника и другие задачи, требующие производительности между классическими MCU и приложенческими процессорами.

\section{Компоненты}
\begin{itemize}
    \item Ядро Arm Cortex-M7 с интегрированным MPU, поддерживающим до 16 зон защиты
    \item 512 КБ объединённой памяти с тесной связью для инструкций и данных (ITCM/DTCM)
    \item 1 МБ встроенной оперативной памяти (OCRAM)
    \item Различные периферийные интерфейсы: 2D графика, интерфейс камеры, контроллеры внешней памяти (SEMC, FlexSPI), аудио интерфейсы, контроллеры дисплея
    \item Контроллер системного сброса (SRC), защищённое энергонезависимое хранилище (SNVS)
    \item Контроллер одноразовой программируемой памяти (OCOTP)
    \item Сетевая шина (NIC-301)
    \item DMA контроллеры (eDMA, DMAMUX)
    \item Продвинутое управление питанием с DCDC и LDO регуляторами
\end{itemize}


\section{Карта памяти}
Карта памяти обширна, включает:
\begin{itemize}
    \item Встроенную оперативную память (OCRAM и FlexRAM)
    \item Память с тесной связью (ITCM/DTCM)
    \item Регистры периферии, отображённые в пространствах AHB/APB
    \item Интерфейсы внешней памяти (FlexSPI, SEMC)
    \item Защищённое энергонезависимое хранилище и область OTP-фьюзов
\end{itemize}

Подробная карта памяти и описание регистров приведены в главах 2, 27, 30, 31 и других разделах руководства.

\section{Механизмы безопасности включают:}
\begin{itemize}
    \item Встроенный MPU с настраиваемыми зонами защиты памяти
    \item Защищённое энергонезависимое хранилище (SNVS) для безопасного хранения ключей и обнаружения взлома
    \item Одноразовые программируемые фьюзы (OTP) для конфигурации загрузки и настроек безопасности
    \item Контроллер системного сброса для управления безопасными последовательностями сброса
    \item Защита доступа в мостах шины (AIPSTZ) для ограничения несанкционированного доступа к периферии и памяти
    \item Конфигурация загрузки через фьюзовую карту, управляющую безопасной загрузкой и блокировкой устройства
    \end{itemize}
Эти механизмы направлены на предотвращение неавторизованного выполнения кода, защиту конфиденциальных данных и обеспечение безопасной загрузки.

\section{Процесс загрузки}
Процесс загрузки гибко настраивается через фьюзы:
\begin{itemize}
    \item Загрузка из внутренней или внешней памяти (FlexSPI, SEMC)
    \item Безопасная загрузка с проверкой подлинности образов с помощью ключей из OTP/SNVS
    \item Фьюзовая карта задаёт выбор загрузочного устройства, параметры безопасности и статус блокировки
    \item Контроллер OTP управляет программированием и блокировкой фьюзов
\end{itemize}
Загрузчик проверяет целостность и подлинность образа перед запуском, используя аппаратные модули безопасности.

\section{Выводы и наблюдения по безопасности}
i.MX RT1060 сочетает высокую производительность с продвинутыми средствами безопасности, подходящими для встроенных систем с повышенными требованиями. Потенциальные риски безопасности:
\begin{itemize}
    \item Очень важно правильно настроить MPU и контролировать доступ, чтобы избежать повышения привилегий
    \item Безопасная загрузка зависит от корректного программирования OTP-фьюзов; ошибки ослабляют защиту
    \item Физические атаки на области OTP и SNVS требуют учёта в модели угроз
    \item Необходимо включать и настраивать защиту шины и периферии для предотвращения несанкционированного доступа
    \end{itemize}
В целом, устройство предоставляет надёжную платформу для безопасных систем, однако безопасность сильно зависит от правильной конфигурации и процессов безопасного программирования.

\begin{thebibliography}{9}
\bibitem{manual}
i.MX RT1060 Processor Reference
Manual, Document Number: IMXRT1060RM
Rev. 3, 07/2021

\bibitem{github}
GitHub Link: https://github.com/MattWay224/reverse-engineering-course

\end{thebibliography}

\end{document}
