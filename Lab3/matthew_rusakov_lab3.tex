\usepackage[utf8]{inputenc}
\usepackage{geometry}
\usepackage[T1]{fontenc}
\usepackage[russian]{babel}
\usepackage{amsmath}

\title{Отчёт по реверс-инжинирингу криптографической функции - Lab3}
\author{Matthew Rusakov m.rusakov@innopolis.university SD-03}
\date{May 2025}


\begin{document}

    \maketitle

    \section*{Предисловие}

    Я проанализировал бинарник тремя способами:
    \begin{itemize}
        \item Статический + динамический анализ (динамический анализ не выдал результатов)
        \item Анализ дизасемблированных функций в ghidra
        \item Анализ бинарника с помощью доп. утилит
    \end{itemize}


    \section{Анализ с помощью команд в терминале (статический анализ)}

    Команда
    \begin{verbatim}
    strings 2 | grep -iE "aes|rsa|des|encrypt|decrypt|crypto|sha|md5"
    \end{verbatim}
    не дала никакого аутпута

    Команда
    \begin{verbatim}
    rabin2 -i 2
    \end{verbatim}
    вывела
    \begin{verbatim}
    # Импорты
[Imports]
nth vaddr      bind   type   lib name
―――――――――――――――――――――――――――――――――――――
1   0x00000000 GLOBAL FUNC       __libc_start_main
2   0x00000000 WEAK   NOTYPE     _ITM_deregisterTMCloneTable
3   0x00001070 GLOBAL FUNC       puts
4   0x00001080 GLOBAL FUNC       __stack_chk_fail
5   0x00001090 GLOBAL FUNC       memcmp
6   0x00000000 WEAK   NOTYPE     __gmon_start__
7   0x00000000 WEAK   NOTYPE     _ITM_registerTMCloneTable
8   0x00000000 WEAK   FUNC       __cxa_finalize
    \end{verbatim}

    Команда
    \begin{verbatim}
    objdump -x 2 | grep -i openssl
    \end{verbatim}
    не вывела ничего

    \paragraph{Вывод}
    Бинарник не использует стандартные криптографические библиотеки (OpenSSL, libcrypto и т.д.), так как:

    \textbf{strings} не нашёл упоминаний AES, RSA, DES, SHA и т.д.

    \textbf{rabin2 -i} не показал импортов криптографических функций.

    \textbf{objdump} тоже не нашёл ссылок на OpenSSL

    Это значит, что шифрование, скорее всего:
    \begin{itemize}
        \item Самописное (например, простой XOR, ROT, замены).
        \item Встроенное в код (без внешних библиотек).
        \item Использует базовые алгоритмы (CRC, Base64, кастомные шифры).
    \end{itemize}

    Для более глубокого анализа я решил дизасемблировать бинарник в ghidra


    \section{Анализ дизассемблированных функций в Ghidra}

    В ходе анализа было исследовано несколько функций, извлечённых с помощью дизассемблирования бинарного файла в Ghidra.
    Цель — определить структуру, поведение и назначение данных функций.
    Основное внимание уделяется функциям, связанным с криптографической обработкой данных.
    Как показал анализ, исследуемый код представляет собой собственную реализацию блочного шифра, архитектурно схожую с AES (Advanced Encryption Standard).

    \subsection{Функция \texttt{FUN\_00101caf}}

    Основная управляющая функция, выполняющая следующие шаги:
    \begin{enumerate}
        \item Инициализация двух массивов: \texttt{local\_a0} и \texttt{local\_60}, содержащих по 16 слов (64 байта).
        \item Инициализация ключевого материала (\texttt{local\_c0}, \texttt{local\_d0} и др.).
        \item Вызов функции \texttt{FUN\_0010167a} — установка начального ключа.
        \item Вызов функции \texttt{FUN\_0010177e} — основной криптографический процесс.
        \item Сравнение результата с эталоном (\texttt{memcmp}).
    \end{enumerate}

    Выводится сообщение \texttt{SUCCESS} или \texttt{FAILURE} в зависимости от результата.

    \subsection{Функция \texttt{FUN\_0010167a}}

    Функция копирует 128-битный ключ из \texttt{param\_3} в область памяти по смещению \texttt{+0xf0} внутри структуры состояния.
    Используется для установки внутреннего состояния перед шифрованием.


    \section{Функция \texttt{FUN\_0010177e}}

    Реализует цикл побайтового шифрования данных, аналогичный \texttt{AES CTR mode}:
    \begin{itemize}
        \item Генерирует ключевой поток путём вызова \texttt{FUN\_001012d2}.
        \item Инкрементирует счётчик (128-битный nonce/counter).
        \item XOR-ит ключевой поток с входными данными.
    \end{itemize}

    Алгоритм работает с блоками по 16 байт, каждый байт обрабатывается в цикле с шагом \texttt{XOR}.

    \subsection{Функция \texttt{FUN\_001012d2}}

    Реализует один раунд симметричного блочного шифра:
    \begin{enumerate}
        \item Применяет подстановку байтов (S-box), используя таблицу \texttt{DAT\_00102140}.
        \item Производит перестановку байтов (аналог ShiftRows в AES).
        \item Выполняет линейную трансформацию, аналогичную \texttt{MixColumns}.
        \item Добавляет раундовый ключ через функцию \texttt{FUN\_0010128c}.
        \item Повторяет 14 раундов, аналогично AES-128.
    \end{enumerate}

    \subsection{Функция \texttt{FUN\_0010128c}}

    Реализация \texttt{AddRoundKey}, то есть побайтового \texttt{XOR} между состоянием (16 байт) и 16-байтным раундовым ключом из расписания.
    Индекс раунда задаётся аргументом \texttt{param\_1}.

    \subsection{Функция \texttt{FUN\_001012c3}}

    Реализует умножение байта на 2 в поле $\mathrm{GF}(2^8)$:
    \[
        \text{xtime}(b) =
        \begin{cases}
            b << 1, & \text{если старший бит } b = 0 \\
            (b << 1) \oplus 0x1B, & \text{если старший бит } b = 1
        \end{cases}
    \]

    Данная операция используется в линейной трансформации \texttt{MixColumns}.


    \section{S-box (\texttt{DAT\_00102140})}

    S-box по адресу \texttt{0x00102140} соответствует таблице подстановок, применяемой в AES:
    \begin{itemize}
        \item \texttt{DAT\_00102140[0x00] = 0x63} — соответствует \texttt{AES S-box[0x00]}.
        \item Применяется при преобразовании каждого байта состояния.
    \end{itemize}

    \subsection*{Вывод}

    В результате анализа установлено, что представленные функции реализуют блочный шифр, чрезвычайно схожий с AES:
    \begin{itemize}
        \item Используются фазы: \texttt{SubBytes}, \texttt{ShiftRows}, \texttt{MixColumns}, \texttt{AddRoundKey}.
        \item Применяется S-box, идентичный AES
        \item Алгоритм построен на 14 раундах, как в AES-128.
    \end{itemize}

    Таким образом, можно сделать вывод, что бинарный модуль реализует кастомную или облегчённую версию AES, возможно, с модифицированным раундовым расписанием.


    \section{Использование автоматических утилит}
    Я попробовал установить утилиту binwalk, которая сможет определить, что за механизм шифрования используется в бинарнике

    \begin{verbatim}
    binwalk -B 2
    \end{verbatim}

    аутпут:
    \begin{verbatim}
    DECIMAL       HEXADECIMAL     DESCRIPTION
    --------------------------------------------------------------------------------
    0             0x0             ELF, 64-bit LSB shared object, AMD x86-64, version 1 (SYSV)
    8256          0x2040          AES Inverse S-Box
    8512          0x2140          AES S-Box
    \end{verbatim}

    \begin{itemize}
        \item ELF-файл (64-битный, LSB, для x86-64) — это стандартный формат исполняемых файлов в Linux
        \item Найден обратный S-box AES (AES Inverse S-box), используемый при расшифровке
        \item Найден прямой S-box AES (AES S-box), используемый при шифровании
    \end{itemize}

    \paragraph{Вывод}
    Обнаруженные S-box'ы подтверждают, что в бинарнике реализован AES или AES-подобный шифр.
    Смещения (0x2040 и 0x2140) указывают на начало массивов, которые используются в подстановочных преобразованиях (SubBytes). Это полностью совпадает с тем, что мы ранее увидели в дизассемблированных функциях: S-box по адресу 0x00102140 и его применение в FUN\_001012d2.


    \begin{thebibliography}{1}
        \bibitem{githublink}
        GitHub Link: https://github.com/MattWay224/reverse-engineering-course
        В этом репозитории можно найти все лабы и информацию про каждое задание в каждой лабе
    \end{thebibliography}
\end{document}
