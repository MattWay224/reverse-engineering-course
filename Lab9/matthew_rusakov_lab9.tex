\usepackage{amsmath}
\usepackage{listings}
\usepackage{color}
\usepackage{geometry}
\usepackage{hyperref}
\usepackage[utf8]{inputenc}
\usepackage[T1]{fontenc}
\usepackage[russian]{babel}

\title{Анализ бинарного файла - Lab9}
\author{Matthew Rusakov m.rusakov@innopolis.university SD-03}
\date{May 2025}


\begin{document}

    \maketitle

    \section*{Предисловие}
    Я изучил бинарник с помощью ghidra и нейросетей.
    Бинарник реализуюет проверку серийного номера (serial) на основе имени пользователя.
    Программа принимает два аргумента командной строки: имя пользователя и серийный номер в формате \texttt{XXXX-YYYY-ZZZZ}, где каждая часть интерпретируется как 32-битное шестнадцатеричное число.


    \section{Первая функция: \texttt{FUN\_00101100}}

    Основная функция (\texttt{main}) выполняет следующий алгоритм:

    \begin{itemize}
        \item Проверяет количество аргументов командной строки:
        \begin{itemize}
            \item Если \texttt{argc == 1}, выводит строку помощи.
            \item Если \texttt{argc == 3}, проверяет длину строки серийного номера: должна быть ровно 0x1c (28) символов.
        \end{itemize}
        \item При корректной длине парсит серийный номер формата \texttt{\%x-\%x-\%x} в три числа.
        \item Вызывает функцию \texttt{FUN\_00101470} для проверки соответствия.
    \end{itemize}


    \section{Функция проверки: \texttt{FUN\_00101470}}

    Функция проверяет три значения серийного номера на соответствие значениям, полученным из имени пользователя:

    \begin{enumerate}
        \item Вызывает \texttt{FUN\_00101380}, которая генерирует MD5-хеш из строки (имени пользователя) с дополнительными преобразованиями.
        \item Вызывает \texttt{FUN\_001012b0} дважды, чтобы получить два результата по данным серийного номера и статическим таблицам.
        \item Проверяет, совпадают ли соответствующие значения:
        \begin{itemize}
            \item \texttt{param\_2[0] == local\_18}
            \item \texttt{param\_2[1] == local\_14}
            \item \texttt{param\_2[2] == local\_10}
        \end{itemize}
    \end{enumerate}


    \section{Генерация MD5: \texttt{FUN\_00101380}}

    Данная функция формирует MD5-хеш следующим образом:

    \begin{enumerate}
        \item Проверяет, что длина имени пользователя не превышает 0x80 символов.
        \item Копирует имя в буфер \texttt{acStack\_1a9}, затем реверсирует строку.
        \item Конкатенирует реверсированную строку с оригинальной и сохраняет в \texttt{local\_128}.
        \item Выполняет \texttt{MD5(local\_128, 256)}.
    \end{enumerate}

    \textbf{Итог:} результат хеша используется как база для сравнения с серийным номером.


    \section{Обработка серийного номера: \texttt{FUN\_001012b0}}

    Функция преобразует части серийного номера, используя следующие шаги:

    \begin{itemize}
        \item Обрабатывает входное значение (серийный номер) и данные таблиц (возможно, секретные, находящиеся по адресу \texttt{DAT\_...}).
        \item На каждой итерации вызывает \texttt{FUN\_00101290}, которая считает количество единичных битов в числе.
        \item Используется схема \textbf{битового xor и сдвига}, результатом чего являются два итоговых значения, записываемые в \texttt{param\_1[0]} и \texttt{param\_1[1]}.
    \end{itemize}


    \section{Подсчет битов: \texttt{FUN\_00101290}}

    Обычная функция подсчета установленных битов (popcount) в 32-битном числе.
    Используется для повышения энтропии и смешивания данных.


    \section{Вывод}

    Вся логика программы направлена на проверку того, соответствует ли серийный номер определённому имени пользователя.
    Алгоритм включает:

    \begin{itemize}
        \item Хеширование имени с помощью модифицированного MD5.
        \item Преобразование серийного номера и сравнение с хешом.
        \item Скрытые данные в памяти программы (таблицы \texttt{DAT\_...}) играют ключевую роль.
    \end{itemize}

    \subsection*{Итоговая схема}

    \begin{center}
        \texttt{username} $\xrightarrow{\text{rev+concat}}$ \texttt{MD5} $\rightarrow$ \texttt{target values} \\
        \texttt{serial} $\xrightarrow{\text{split}}$ \texttt{transformation} $\rightarrow$ \texttt{candidate values} \\
        \texttt{if match $\Rightarrow$ OK, else $\Rightarrow$ Try again}
    \end{center}
    \subsection*{Решение?}
    Я попробовал сделать генератор на питоне, который будет создавать value для ключа по такому же формату.
    Я вставлю их в dumps.log в ассемблер виде, а также сам скрипт и массивы байтов вы можете найти на github
    \begin{verbatim}
        m@hp:~/PycharmProjects/reverse-engineering-course/Lab9$ python3 lab_data/generator.py
        Сгенерированный серийный номер для 'admin': 1A50-51E4-4BB4
        Сгенерированный серийный номер для 'testuser': 6A9E-EE5C-84C2
        Сгенерированный серийный номер для 'averylongusernameexample': B7D2-A894-1F46

    \end{verbatim}

    \paragraph{HO}
    я так и не понял, правильные это ключи или нет
    \begin{verbatim}
        m@hp:~/PycharmProjects/reverse-engineering-course/Lab9$ ./9 admin 1A50-51E4-4BB4
        m@hp:~/PycharmProjects/reverse-engineering-course/Lab9$ ./9 matt serial
        m@hp:~/PycharmProjects/reverse-engineering-course/Lab9$
    \end{verbatim}

    \begin{thebibliography}{1}
        \bibitem{githublink}
        GitHub Link: https://github.com/MattWay224/reverse-engineering-course
        В этом репозитории можно найти все лабы и информацию про каждое задание в каждой лабе
    \end{thebibliography}
\end{document}
